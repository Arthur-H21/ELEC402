\documentclass[12pt]{article}
\usepackage[margin=0.9in]{geometry} 
\usepackage[utf8]{inputenc}
\usepackage{amsmath}
\usepackage{amsfonts}
\usepackage{amssymb}
\usepackage{graphicx}
\usepackage{color}
\usepackage{float}
\usepackage{scrextend}
\usepackage{enumitem}

%ignore \hbox issue
\hfuzz=10000pt

% title declarations
\newcommand{\doctitle}{Project 3 - The MOS Transistor}
\newcommand{\docsubtitle}{Noise Margins, VTC and Cadence Simulations}

\renewcommand{\maketitle} {
    \setlength{\parindent}{0pt}
    \begin{center} \
        % top spacing
        \vspace*{1in}

        % main title
        \huge{\doctitle}\\
        \Large{\docsubtitle}

        % naming
        \vspace*{0.2in}
        \large{
            Arthur Hsueh\\
            21582168\\
            UBC - ELEC 402
        }


    \end{center}
}

\begin{document}
\maketitle
\thispagestyle{empty}
\pagebreak

\tableofcontents
\thispagestyle{empty}

\listoffigures
\thispagestyle{empty}
\pagebreak
\setcounter{page}{1}

\section{Designing widths of pull-down transistors}
This problem asks to design the widths of the pull-down transistors such that $V_{OL}$ = 0.1 V. The general
method is to equate $I_{L}$, the current from the load resistor/transsitor, to $I_{DI}$, the drain current of the 
pull-down transistor. Because we are evalulating for the design requirement of $V_{OL}$, we take the input as
$V_{in} = V_{DD}$, and thus the pull-down transistor will always be in the linear regime.

\subsection{Calculations}
\subsection*{Resistive load inverter}
This is the circuit that contains the 10k$\Omega$ resistor. We begin by equating currents,
\[ I_R = I_{D}(linear) \]
\[\frac{V_{DD} - V_{OL}}{R_L} = \frac{W_N}{L_N} \mu_N C_{ox} ((V_{DD} - V_T) V_{OL} - \frac{V_{OL}^2}{2}) \]

Values can be substituted in, (constants used will be taken from the Useful\_Formula.pdf document on canvas)

\[\frac{1.2 - 0.1}{10000} = \frac{W_N}{0.1*10^{-7}}*270*1.6*10^{-6} *((1.2 - 0.4)*0.1 - \frac{0.1^2}{2}) \]
Solving for $W_N$ yields
\[ W_N = 3.395 * 10^{-8}m = 33.95 nm\]

\subsection*{Saturated-enhancement-load inverter}
This is the circuit that contains the pull-up NMOS transistor with 1.2V gate voltage. Here, the pull-up
transistor is operating in the satuation regime ($V_{DS} > V_{GS} - V_T$). Again, we equate curents to solve,
\[ I_{S pull-up}(saturation) = I_{D}(linear)\]
\[\frac{W_L v_{sat}C_{ox}(V_{DD} - V_{OL} - V_{TL})^2}{(V_{DD} - V_{OL} - V_{TL}) + E_C L_L} = \frac{W_N}{L_N} \mu_N C_{ox} ((V_{DD} - V_T)V_{OL} - \frac{V_{OL}^2}{2}) \]
Substituting values,
\[\frac{0.1*10^{-5}* 8*10^6 *1.6 * 10^{-6}(1.2 - 0.1 - 0.4)^2}{(1.2 - 0.1 - 0.4) + 6*0.1*10^{-7}} = \frac{W_N}{0.1*10^{-7}}*270*1.6*10^{-6} *((1.2 - 0.4)*0.1 - \frac{0.1^2}{2})\]
Solving for $W_N$ yields
\[ W_N = 2.765*10^-9m = 2.765nm\]

\subsection*{'Linear'-load inverter}
This is the circuit that contains the pull-up NMOS transistor with 1.6V gate voltage. Here, the pull-up transistor
is operating in the linear regime ($V_{DS} < V_{GS} - V_T$). This calculation allows for cancellation of many terms
Again, we equate currents to solve,
\[ I_{S pull-up}(linear) = I_{D}(linear)\]
Cancelling identical terms yields,
\[W_L ((V_{gate} - V_{OL} - V_T) V_{OL} - \frac{V_{OL}^2}{2}) = W_N ((V_{DD} - V_T) V_{OL} - \frac{V_{OL}^2}{2}) \]
Substituting values,
\[0.1*10^{-7} ((1.6 - 0.1 - 0.4)*0.1 - \frac{0.1^2}{2}) = W_N ((1.2 - 0.4)*0.1 - \frac{0.1^2}{2}) \]
Solving for $W_N$ yields 
\[ W_N = 1.4*10^{-8} = 14nm\]

\subsection{Results}
testing tesgin 2
\pagebreak

\end{document}